\documentclass[12pt]{beamer}
\usepackage[utf8]{inputenc}
\usepackage[T1]{fontenc}
\usepackage[french]{babel}
\usepackage{amsmath,amssymb,tabularx}
\usepackage{graphicx}
\usetheme{jambro}

\title[STT 504 Sondage]{Estimation de la variance de statistiques non - linéaires dans des plans complexes}
\author{FM, Université Soumaré}
\date{}

\begin{document}

\begin{frame}
\titlepage
\end{frame}

\begin{frame}{Sommaire}
\tableofcontents
\end{frame}

% ==============================================
\section{8.0 Contexte et Motivation}
% ==============================================
\begin{frame}{8.1 Contexte et motivation}
\small
\textbf{Problématique :} Au delà des estimateurs naturels, il est difficile de calculer / compute les variances des différents estimateurs.  

\textbf{Solution :} Application d'approches provenant d'autres domaines des statistiques ou des mathematiques :
\begin{itemize}
    \item Mise à profit de la formule de Taylor
    \item Utilisation de la randomisation,
    \item Utilisation des méthodes Bootstrap ou Jackknife provenant de l'estimation mathématique.
\end{itemize}

\begin{alertblock}{Avantages ET Inconvénients}
\begin{itemize}
    \item + Possibilité de calculer des variances dans des plans de sondage complexes
    \item - Sous certaines conditions, les variances produites par ces méthodes sont plus grandes que la variance théorique attendue. 
\end{itemize}
\end{alertblock}

\end{frame}

% ==============================================
\section{8.2 Méthode de linéarisation}
% ==============================================
% ==============================================
\begin{frame}{8.2.1 Principe de base}
% ==============================================
\small
L'Approche par linéarisation s'appuie sur la décomposition des fonctions en serie de Taylor.

Soit l'objectif suivant : estimer un parametre qui est une fonction non-lineaire
 de totaux, $\theta = f(t_Y,t_X) $ et ensuite calculer ou approcher sa variance.
\end{frame}
% ==============================================
\begin{frame}{7.2.2 Généralisation}
% ==============================================
\end{frame}
% ==============================================
\begin{frame}{7.2.3 Cas d'usage}
% ==============================================
\end{frame}
% ==============================================
\section{7.3 Méthode des groupes aléatoire}
% ==============================================
\begin{frame}{7.3.1 Tirage à probabilités égales}
\scriptsize
\begin{columns}
\begin{column}{0.6\textwidth}
\[
\hat{T} = \sum_{h=1}^m \sum_{i=1}^{n_h} \frac{MN_h}{mn_h}y_{hi}
\]
\begin{itemize}
    \item UP tirées avec $\pi_h = m/M$
    \item US tirées avec $\pi_{i|h} = n_h/N_h$
\end{itemize}
\end{column}
\begin{column}{0.4\textwidth}
\begin{exampleblock}{Cas auto-pondéré}
Si $n_h = n_0$ constant et $\pi_h \propto N_h$ :
\[
\hat{T} = \frac{N}{mn_0}\sum y_{hi} \quad \text{(Moyenne simple)}
\]
\end{exampleblock}
\end{column}
\end{columns}

\begin{alertblock}{Attention !}
Adapter $n_h$ si $N_h' \neq N_h$ (tailles réelles vs théoriques) :
\[
n_{0h}' = \frac{N_h'}{N_h}n_0
\]
\end{alertblock}
\end{frame}

% ==============================================
\section{7.4 Effet de grappe}
% ==============================================
\begin{frame}{7.4 Impact de la corrélation intra-UP}
\footnotesize
\textbf{Coefficient de corrélation intra-grappe :}
\[
\rho = \frac{\sum_{h=1}^M \sum_{i\neq j}(Y_{hi}-\bar{Y})(Y_{hj}-\bar{Y})}{(N-1)S^2}
\]

\begin{block}{Effet sur la variance}
Pour UP de taille constante $N$ et $n$ US/UP :
\[
V(\hat{T}) = N^2\frac{S^2}{mn}\left(1 + \rho(n-1)\right)
\]
\end{block}

\begin{itemize}
    \item $\rho > 0$ : Similarité des réponses dans une UP
    \item \textbf{Deff} = $1 + \rho(n-1)$ : Facteur d'inflation de variance
\end{itemize}

\begin{exampleblock}{Dimensionnement d'échantillon}
\[
n \geq \frac{1.96^2 P(1-P)}{\varepsilon^2(1-t)} \times \text{Deff} \quad (\text{Deff} \approx 1.5 \text{ par défaut})
\]
\end{exampleblock}
\end{frame}

% ==============================================
\section{7.5 Optimisation budgétaire}
% ==============================================
\begin{frame}{7.5 Allocation optimale des ressources}
\scriptsize
\textbf{Fonction de coût :} 
\[
C_T = c_1m + c_2mn \quad \begin{cases}
c_1 : \text{Coût fixe/UP} \\
c_2 : \text{Coût variable/US}
\end{cases}
\]

\textbf{Solution optimale :}
\[
\bar{n} = \sqrt{\frac{c_1}{c_2} \cdot \frac{1-\rho}{\rho}} \quad ; \quad m = \frac{C_T}{c_1 + c_2\bar{n}}
\]

\begin{block}{Stratégie pratique}
\begin{enumerate}
    \item Estimer $\rho$ via des enquêtes pilotes
    \item Calculer $\bar{n}$ pour équilibrer variance et coût
    \item Ajuster $m$ selon le budget disponible
\end{enumerate}
\end{block}

\begin{alertblock}{Recommendation}
Privilégier plus d'UP avec moins d'US/UP quand $\rho$ est élevé.
\end{alertblock}
\end{frame}

\begin{frame}{Synthèse des concepts clés}
\begin{itemize}
    \item[$\blacktriangleright$] Hiérarchie UP/US pour contourner l'absence de base de sondage
    \item[$\blacktriangleright$] Double composante de variance (inter/intra-UP)
    \item[$\blacktriangleright$] Impact critique de l'effet de grappe ($\rho$)
    \item[$\blacktriangleright$] Optimisation budget-précision via $\bar{n}$ et $m$
\end{itemize}

\vspace{0.5cm}
\begin{block}{Perspective}
Intégration avec la stratification et les probabilités inégales pour des plans complexes.
\end{block}
\end{frame}

\end{document}