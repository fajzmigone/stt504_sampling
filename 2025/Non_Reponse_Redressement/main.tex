\documentclass[12pt]{beamer}
\usepackage[utf8]{inputenc}
\usepackage[T1]{fontenc}
\usepackage[french]{babel}
\usepackage{amsmath,amssymb,tabularx,booktabs}
\usepackage{graphicx}
\usetheme{jambro}

\title[STT 504 Sondage]{Non Reponse et Redressement}
\author{FM, Université Soumaré}
\date{}

\begin{document}

\begin{frame}
\titlepage
\end{frame}

\begin{frame}{Sommaire}
\tableofcontents
\end{frame}

% ==============================================
\section{9.1 Principes des redressements}
% ==============================================
\begin{frame}{9.1.1 Contexte et justification}
\begin{itemize}
\item Problème courant : Biais d'échantillonnage malgré un tirage probabiliste
\item Solution : Utiliser des \textbf{variables auxiliaires} connues sur la population
\item Objectifs :
\begin{itemize}
\item Améliorer la précision des estimateurs
\item Corriger les déséquilibres structurels
\item Compenser les non-réponses
\end{itemize}
\end{itemize}

\begin{exampleblock}{Exemple concret}
Enquête sur le revenu où l'âge moyen échantillonné diffère de la population. Utiliser la répartition démographique officielle pour redresser.
\end{exampleblock}
\end{frame}

% ==============================================
\section{9.2 Méthodes de redressement}
% ==============================================
\begin{frame}{9.2.1 Post-stratification simple}
\scriptsize
\begin{columns}[T]
\begin{column}{0.6\textwidth}
\textbf{Estimateur post-stratifié :}
\[
\hat{\bar{Y}}_{post} = \frac{1}{N}\sum_{h=1}^H N_h \bar{y}_h
\]
\begin{itemize}
\item $N_h$ : Taille connue des strates
\item $\bar{y}_h$ : Moyenne échantillonnale par strate
\end{itemize}

\textbf{Variance approximée :}
\[
V(\hat{\bar{Y}}_{post}) \approx \frac{(1-f)}{n}\left(\sum \frac{N_h}{N}S_h^2\right) + \frac{1-f}{n^2}\left(\sum (1-\frac{N_h}{N})S_h^2\right)
\]
\end{column}

\begin{column}{0.4\textwidth}
\begin{alertblock}{Avantages/Inconvénients}
\begin{itemize}
\item[+] Simple à implémenter
\item[+] Réduit le biais
\item[--] Variance légèrement supérieure à la stratification classique
\end{itemize}
\end{alertblock}
\end{column}
\end{columns}
\end{frame}

\begin{frame}{9.2.2 Raking Ratio (Calage sur marges)}
\footnotesize
\textbf{Objectif :} Aligner les marges d'un tableau croisé échantillon-population

\begin{exampleblock}{Algorithme itératif}
\begin{enumerate}
\item Caler les lignes : $n_{ij}^{(1)} = n_{ij} \frac{N_i}{n_i}$
\item Caler les colonnes : $n_{ij}^{(2)} = n_{ij}^{(1)} \frac{N_j}{n_j^{(1)}}$
\item Répéter jusqu'à convergence
\end{enumerate}
\end{exampleblock}

\begin{equation*}
\begin{cases}
\sum_j n_{ij} = N_i \\
\sum_i n_{ij} = N_j 
\end{cases}
\end{equation*}

\textbf{Application :} Enquêtes électorales avec calage sur âge/sexe/région
\end{frame}

% ==============================================
\section{9.3 Méthodes paramétriques}
% ==============================================
\begin{frame}{9.3.1 Estimateur par ratio}
\scriptsize
\textbf{Hypothèse :} Relation linéaire $Y \approx rX$

\[
\hat{\bar{Y}}_{ratio} = \frac{\bar{X}}{\bar{x}} \bar{y} \quad \text{avec } \bar{X} \text{ connu}
\]

\textbf{Précision améliorée si :}
\[
\rho > \frac{1}{2}\frac{CV(X)}{CV(Y)}
\]

\begin{exampleblock}{Cas pratique}
Estimation de la production agricole ($Y$) à partir de la superficie cultivée ($X$)
\end{exampleblock}
\end{frame}

\begin{frame}{9.3.2 Régression linéaire}
\footnotesize
\textbf{Modèle :} $y_i = a + b x_i + \epsilon_i$

\[
\hat{\bar{Y}}_{reg} = \bar{y} + \hat{b}(\bar{X} - \bar{x})
\]

\textbf{Propriétés :}
\begin{itemize}
\item Biais d'ordre $1/n$
\item Variance réduite : $V(\hat{\bar{Y}}_{reg}) \approx V(\hat{\bar{Y}})(1-\rho^2)$
\end{itemize}

\begin{alertblock}{Application}
Correction de biais socio-démographiques dans les enquêtes santé
\end{alertblock}
\end{frame}

% ==============================================
\section{9.4 Traitement des non-réponses}
% ==============================================
\begin{frame}{9.4.1 Typologie des non-réponses}
\begin{columns}[T]
\begin{column}{0.5\textwidth}
\textbf{Non-réponse totale :}
\begin{itemize}
\item Aucune donnée collectée
\item Causes : Absence, refus, erreur logistique
\item Correction : Repondération
\end{itemize}
\end{column}

\begin{column}{0.5\textwidth}
\textbf{Non-réponse partielle :}
\begin{itemize}
\item Données incomplètes
\item Causes : Oubli, sensibilité
\item Correction : Imputation
\end{itemize}
\end{column}
\end{columns}

\begin{block}{Règle empirique}
Taux > 40\% $\Rightarrow$ Remettre en cause la validité de l'enquête
\end{block}
\end{frame}

\begin{frame}{9.4.2 Méthodes de repondération}
\scriptsize
\textbf{Estimation des probabilités de réponse $R_i$ :}
\[
w'_i = \frac{w_i}{\hat{R}_i} \quad \text{avec } w_i = 1/\pi_i
\]

\begin{itemize}
\item \textbf{Regroupement :} $\hat{R}_h = \frac{n_{h,répondants}}{n_h}$
\item \textbf{Modèle logistique :} $R_i = \frac{e^{\beta X_i}}{1 + e^{\beta X_i}}$
\end{itemize}

\begin{equation*}
\hat{\bar{Y}} = \frac{1}{N}\sum_{h=1}^H \frac{n_h}{n_{h,répondants}} \sum_{i \in h} y_i
\end{equation*}
\end{frame}

\begin{frame}{9.4.3 Techniques d'imputation}
\scriptsize
\begin{tabular}{|l|l|l|}
\toprule
\textbf{Méthode} & \textbf{Principe} & \textbf{Usage} \\
\midrule
Hot-deck & Donneur similaire & Enquêtes ménages \\
Cold-deck & Source externe & Séries temporelles \\
Régression & Modèle prédictif & Données économiques \\
Moyenne locale & Voisinage & Données spatiales \\
\bottomrule
\end{tabular}

\begin{exampleblock}{Exemple d'imputation multiple}
Utiliser MICE (Multiple Imputation by Chained Equations) pour des jeux de données complexes
\end{exampleblock}
\end{frame}

% ==============================================
\section*{Conclusion}
% ==============================================
\begin{frame}{Synthèse des bonnes pratiques}
\begin{itemize}
\item[$\bullet$] Valider systématiquement les hypothèses de redressement
\item[$\bullet$] Documenter rigoureusement les taux de non-réponse
\item[$\bullet$] Privilégier les méthodes transparentes et reproductibles
\item[$\bullet$] Combiner méthodes paramétriques et non-paramétriques
\end{itemize}

\begin{block}{Perspectives}
Intégration croissante de l'apprentissage automatique pour l'imputation avancée
\end{block}
\end{frame}

\end{document}