\documentclass{beamer}

\usetheme{jambro}  % Thème de la présentation
\usecolortheme{default}  % Thème de couleur

\title{Chapitre 1 : Généralités des Enquêtes par Sondage}
\author{Franck MIGONE}
\institute{Université Soumaré}
\date{\today}

\begin{document}

\begin{frame}
  \titlepage
\end{frame}

\begin{frame}
  \frametitle{Introduction}
  \begin{itemize}
    \item Une enquête par sondage est une technique d’observation qui permet de produire de l’information statistique sur une population donnée à partir de l’observation d’une partie seulement de cette population.
    \item Elle se justifie particulièrement dans l’étude de populations très nombreuses ou très dispersées.
    \item Elle peut remplacer utilement les recensements, souvent très lourds et très onéreux.
  \end{itemize}
\end{frame}

\begin{frame}
  \frametitle{Utilisation des Sondages}
  \begin{itemize}
    \item Dans les pays développés, les sondages sont très couramment utilisés avant chaque élection importante.
    \item Les instituts de sondage présentent des prévisions des résultats du vote.
    \item Les sondages « sortie des urnes » sont souvent très proches de la réalité.
  \end{itemize}
\end{frame}

\begin{frame}
  \frametitle{Exemple Historique}
  \begin{itemize}
    \item En 1936, lors des élections présidentielles américaines, le candidat républicain était donné gagnant par une large majorité, mais il obtint finalement moins de 40\% des suffrages.
    \item Le sondage était basé sur une liste tirée de l’annuaire téléphonique et parmi les lecteurs du quotidien « Literacy Digest », très favorable aux républicains.
    \item De telles erreurs sont aujourd’hui évitées grâce à des méthodes plus rigoureuses.
  \end{itemize}
\end{frame}

\begin{frame}
  \frametitle{Questions Clés pour les Statisticiens}
  \begin{itemize}
    \item Quelle est la meilleure méthode de sondage à utiliser ?
    \item Combien d’individus faut-il enquêter pour avoir des résultats généralisables ?
    \item Comment constituer l’échantillon ?
    \item Comment agréger les données pour avoir la meilleure information ?
    \item Quelle est la variabilité de cet estimateur ?
  \end{itemize}
\end{frame}

\begin{frame}
  \frametitle{Domaines d'Application des Sondages}
  \begin{itemize}
    \item La médecine (essais cliniques)
    \item La biologie (prévalence d’une IST)
    \item La recherche de gisements pétroliers ou miniers
    \item L’agriculture (évaluation du volume de la production agricole nationale)
    \item La vérification d’une comptabilité
    \item Les contrôles fiscaux
    \item Les contrôles de la qualité des produits sur les chaînes d’usines
    \item Le contrôle de la qualité d’un recensement (enquête de couverture)
    \item Le contrôle anti-dopage en sport
    \item Les préparations culinaires
  \end{itemize}
\end{frame}

\begin{frame}
  \frametitle{Erreurs d’Observation et Erreur de Sondage}
  \begin{itemize}
    \item Les données collectées sont toujours entachées d’erreur.
    \item Deux types d’erreurs : l’erreur d’observation et l’erreur d’échantillonnage.
    \item Un recensement induit uniquement une erreur d’observation.
    \item Une enquête par sondage induit à la fois une erreur d’observation et une erreur d’échantillonnage.
  \end{itemize}
\end{frame}

\begin{frame}
  \frametitle{Notions Statistiques et Vocabulaire de Base}
  \begin{itemize}
    \item \textbf{Individu} : Unité statistique de base.
    \item \textbf{Population} : Ensemble sur lequel porte l’étude statistique.
    \item \textbf{Échantillon} : Sous-ensemble de la population.
    \item \textbf{Base de sondage} : Liste complète des individus de l’univers statistique.
    \item \textbf{Variable d’intérêt} : Variable qui formalise l’information objet du sondage.
    \item \textbf{Estimateur et estimation} : Outil mathématique pour approximer une grandeur.
    \item \textbf{Biais} : Écart entre l’espérance de l’estimateur et la vraie valeur.
    \item \textbf{Plan de sondage} : Document définissant les modalités du tirage de l’échantillon.
  \end{itemize}
\end{frame}

\end{document}