\documentclass{beamer}
\usepackage[utf8]{inputenc}
\usepackage{graphicx}
\usepackage{lmodern}
\usetheme{jambro}

\title[Chapitre 2]{Chapitre 2 : Les méthodes empiriques d’échantillonnage - Master 2 STT}
\author{Franck MIGONE}
\date{}

\begin{document}

\frame{\titlepage}

\section{Introduction}

\begin{frame}{Introduction}
\begin{itemize}
    \item Les méthodes empiriques d'échantillonnage sont aussi appelées \textbf{méthodes de sondages à choix raisonné}.
    \item Elles permettent de rendre un échantillon représentatif même en l'absence d'information complète.
    \item Utilisées lorsque :
    \begin{itemize}
        \item Les erreurs d’échantillonnage ne sont pas une préoccupation majeure.
        \item Les ressources financières sont limitées.
    \end{itemize}
\end{itemize}
\end{frame}

\section{Les différentes méthodes}

\subsection{Méthode des quotas}

\begin{frame}{Méthode des quotas}
\begin{itemize}
    \item Reproduit à échelle réduite la structure de l'univers statistique.
    \item Nécessite la connaissance précise de la structure de l'univers.
    \item Exemple :
    \begin{itemize}
        \item Étudier le salaire moyen des travailleurs selon le sexe, l'ancienneté, et la catégorie socioprofessionnelle.
    \end{itemize}
    \item Calcul des proportions des sous-populations pour guider les quotas.
\end{itemize}
\end{frame}

\begin{frame}{Exemple d'application des quotas}
\begin{itemize}
    \item Total femmes : $106000$, total hommes : $86500$.
    \item Femmes dans l'échantillon ($55,1\%$) : $551$.
    \item Hommes dans l'échantillon ($44,9\%$) : $449$.
\end{itemize}
\begin{table}[]
    \centering
    \caption{Répartition des femmes dans l'échantillon}
    \begin{tabular}{|c|c|c|c|c|}
        \hline
        Catégorie & A & B & C & D \\
        \hline
        0-5 ans & 8 & 26 & 104 & 52 \\
        6-10 ans & 16 & 78 & 78 & 78 \\
        11 ans et plus & 26 & 39 & 18 & 29 \\
        \hline
        Total & 49 & 143 & 200 & 158 \\
        \hline
    \end{tabular}
\end{table}
\end{frame}

\subsection{Méthode des unités-types}

\begin{frame}{Méthode des unités-types}
\begin{itemize}
    \item Subdivise la population en sous-populations homogènes.
    \item Forme l'échantillon en choisissant des unités-types dans chaque sous-population.
    \item Avantages :
    \begin{itemize}
        \item Rapidité.
        \item Précision accrue pour les populations homogènes.
    \end{itemize}
    \item Extrapolation nécessite une connaissance précise des effectifs des groupes.
\end{itemize}
\end{frame}

\subsection{Méthode des itinéraires}

\begin{frame}{Méthode des itinéraires}
\begin{itemize}
    \item L'enquêteur suit des itinéraires précis et s'arrête à des lieux prédéterminés.
    \item Avantages :
    \begin{itemize}
        \item Prend en compte les caractéristiques géographiques.
        \item Coût réduit.
    \end{itemize}
\end{itemize}
\end{frame}

\subsection{Méthode "boule de neige"}

\begin{frame}{Méthode "boule de neige"}
\begin{itemize}
    \item Identifie un individu possédant une caractéristique rare.
    \item Celui-ci indique d'autres individus similaires.
    \item Avantages :
    \begin{itemize}
        \item Utile pour les populations difficiles à localiser.
    \end{itemize}
    \item Inconvénients :
    \begin{itemize}
        \item Biais de similarité.
        \item Risque de double comptage.
    \end{itemize}
\end{itemize}
\end{frame}

\subsection{Enquêtes par téléphone ou email}

\begin{frame}{Enquêtes par téléphone ou email}
\begin{itemize}
    \item Enquêtes anonymes avec coûts réduits.
    \item Contraintes :
    \begin{itemize}
        \item Représentativité limitée dans certains contextes (taux d'équipement).
    \end{itemize}
    \item Promesses futures :
    \begin{itemize}
        \item Téléphones mobiles en Afrique.
    \end{itemize}
\end{itemize}
\end{frame}

\subsection{Enquêtes par volontariat}

\begin{frame}{Enquêtes par volontariat}
\begin{itemize}
    \item Sollicitation d'individus prêts à participer spontanément à l'enquête.
    \item Avantages :
    \begin{itemize}
        \item Facilité et coût réduit.
    \end{itemize}
    \item Inconvénients :
    \begin{itemize}
        \item Biais de sélection dû à la motivation des participants.
    \end{itemize}
\end{itemize}
\end{frame}

\section{Conclusion}

\begin{frame}{Conclusion}
\begin{itemize}
    \item Les méthodes empiriques sont essentielles dans des contextes où les ressources sont limitées.
    \item Chaque méthode présente des avantages et des limites spécifiques :
    \begin{itemize}
        \item Les quotas offrent une structure représentative basée sur des proportions précises.
        \item Les unités-types et itinéraires permettent d'adapter l'échantillon à des contextes géographiques ou homogènes.
        \item Les enquêtes par téléphone, email ou volontariat apportent flexibilité mais peuvent souffrir de biais.
    \end{itemize}
    \item Un choix adapté au contexte garantit une meilleure représentativité et pertinence des résultats.
\end{itemize}
\end{frame}

\end{document}
