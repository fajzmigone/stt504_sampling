\documentclass{beamer}
\usetheme{jambro}
\usepackage[utf8]{inputenc} % Recommended for accents
\usepackage[T1]{fontenc}    % Recommended for font encoding
% Pour une meilleure typographie française (dates, césures, etc.), décommentez la ligne suivante :
% \usepackage[french]{babel}
\usepackage{amsmath}
\usepackage{booktabs}
\usepackage{multirow}
\usepackage{graphicx}

\title{Analyse de Données Catégorielles dans les Enquêtes Complexes}
\subtitle{Effets du Plan de Sondage sur les Tests du Chi-Carré et les Intervalles de Confiance}
\author[FM]{FM \\ \small} % Placeholders
\date{\today}
\institute[Université Soumaré]{Université Soumaré} % Placeholder

\begin{document}

\frame{\titlepage}

% Introduction
\section{Introduction}
\begin{frame}{Introduction}
\frametitle{Objectifs et Sujets Clés}
\begin{block}{Objectifs}
\begin{itemize}
\item Comprendre l'impact des plans de sondage complexes sur l'analyse de données catégorielles
\item Apprendre les procédures correctes pour les tests du chi-carré dans les enquêtes complexes
\item Reconnaître les conséquences de l'ignorance du plan de sondage
\end{itemize}
\end{block}

\begin{block}{Sujets Clés}
\begin{itemize}
\item Revue des tests du chi-carré sous échantillonnage aléatoire simple (SAS)
\item Effets de la stratification et de l'échantillonnage en grappes
\item Estimation pondérée dans les tableaux de contingence
\item Implications pratiques et études de cas
\end{itemize}
\end{block}
\end{frame}

% Contexte Théorique
\section{Contexte Théorique}
\begin{frame}{Tests du Chi-Carré sous SAS}
\frametitle{Cadre d'Échantillonnage Multinomial}
\begin{itemize}
\item \textbf{Hypothèses}:
\begin{itemize}
\item Observations indépendantes
\item Grande taille d'échantillon ($n \geq 5 \times$ nombre de cellules)
\item Effectifs attendus par cellule $> 1$
\end{itemize}

\item \textbf{Tests Courants}:
\begin{itemize}
\item Test d'indépendance : $H_0: p_{ij} = p_{i+}p_{+j}$
\item Test d'homogénéité : $H_0: p_{1j} = p_{2j} = \cdots = p_{rj}$
\item Test d'ajustement : $H_0: p_i = p_i^{(0)}$
\end{itemize}
\end{itemize}

\begin{exampleblock}{Exemple 10.1}
500 couples mariés interrogés sur la possession d'ordinateur et l'abonnement au câble :
\begin{center}
\begin{tabular}{lcc}
\toprule
 & Ordinateur Oui & Ordinateur Non \\
\midrule
Câble Oui & 119 & 188 \\
Câble Non & 88 & 105 \\
\bottomrule
\end{tabular}
\end{center}
\end{exampleblock}
\end{frame}

% Statistiques de Test
\begin{frame}{Statistiques de Test}
\frametitle{Statistiques de Pearson et du Rapport de Vraisemblance}
\begin{columns}
\begin{column}{0.5\textwidth}
\textbf{$X^2$ de Pearson}:
\[ X^2 = \sum_{i=1}^r \sum_{j=1}^c \frac{(O_{ij} - E_{ij})^2}{E_{ij}} \]
où :
\begin{itemize}
\item $O_{ij}$ = effectif observé
\item $E_{ij} = n\hat{p}_{i+}\hat{p}_{+j}$
\end{itemize}
\end{column}

\begin{column}{0.5\textwidth}
\textbf{Rapport de Vraisemblance $G^2$}:
\[ G^2 = 2\sum_{i=1}^r \sum_{j=1}^c O_{ij} \ln\left(\frac{O_{ij}}{E_{ij}}\right) \]
\end{column}
\end{columns}

\begin{block}{Résultats de l'Exemple}
\begin{itemize}
\item $X^2 = 2.281$, $G^2 = 2.275$
\item $p$-valeur = 0.13 (non-rejet de $H_0$)
\item Rapport de cotes = 0.755 (IC 95\% inclut 1)
\end{itemize}
\end{block}
\end{frame}

% Effets du Plan de Sondage
\section{Effets du Plan de Sondage}
\begin{frame}{Effets de l'Échantillonnage en Grappes}
\frametitle{Impact de la Corrélation Intra-Classe (CIC)}
\begin{itemize}
\item \textbf{Problème}: CIC positive $\Rightarrow$ observations dépendantes
\item \textbf{Conséquence}: Statistiques de test gonflées par l'effet du plan de sondage (deff)
\[ \text{deff} = 1 + (m - 1)\rho \]
où $m$ = taille de grappe, $\rho$ = CIC
\end{itemize}

\begin{exampleblock}{Exemple 10.4}
\begin{itemize}
\item Paires mari-femme (CIC = 1)
\item Ignorer l'effet de grappe double les statistiques de test :
\begin{itemize}
\item $X^2$ augmente de 2.281 à 4.562
\item $p$-valeur incorrecte = 0.033 (Erreur de type I)
\end{itemize}
\end{itemize}
\end{exampleblock}

\begin{alertblock}{Point Clé}
La vraie distribution sous $H_0$ est $X^2/\text{deff} \sim \chi^2$
\end{alertblock}
\end{frame}

\begin{frame}{Effets de la Stratification}
\frametitle{Gains de Précision et Tests Conservateurs}
\begin{columns}
\begin{column}{0.6\textwidth}
\begin{itemize}
\item \textbf{Avantage}: Précision accrue (deff < 1)
\item \textbf{Impact sur le Test}:
\begin{itemize}
\item Les $p$-valeurs basées sur l'SAS deviennent conservatives
\item $\alpha$ réel < $\alpha$ nominal
\end{itemize}
\item \textbf{Exemple}: Analyse NCVS
\begin{itemize}
\item $p$-valeur rapportée : 0.04
\item $p$-valeur réelle : 0.02
\end{itemize}
\end{itemize}
\end{column}

\begin{column}{0.4\textwidth}
\begin{center}
% Le nom du fichier image reste inchangé
\includegraphics[width=\textwidth]{stratification_effect.png}
\end{center}
\end{column}
\end{columns}

\begin{block}{Recommandation}
\begin{itemize}
\item Pour une inférence valide, incorporer la stratification dans l'analyse
\item Utiliser des logiciels spécialisés (ex: SUDAAN, package R \texttt{survey})
\end{itemize}
\end{block}
\end{frame}

% Analyse Pondérée
\section{Analyse Pondérée}
\begin{frame}{Estimation Pondérée}
\frametitle{Tableaux de Contingence dans les Enquêtes Complexes}
\begin{block}{Estimateur de Proportion Pondéré}
\[ \hat{p}_{ij} = \frac{\sum_{k \in S} w_k y_{kij}}{\sum_{k \in S} w_k} \]
où :
\begin{itemize}
\item $w_k$ = poids de sondage
\item $y_{kij} = \mathbb{I}\{\text{unité } k \text{ dans cellule } (i,j)\}$
\end{itemize}
\end{block}

\begin{exampleblock}{Considérations Pratiques}
\begin{itemize}
\item Comparer les rapports de cotes pondérés vs. non pondérés
\item Investiguer les divergences (peuvent indiquer des effets du plan)
\item Utiliser la linéarisation de Taylor ou des méthodes de réplication pour l'estimation de variance
\end{itemize}
\end{exampleblock}
\end{frame}

% Implications Pratiques
\section{Implications Pratiques}
\begin{frame}{Conséquences de l'Ignorance du Plan}
\frametitle{Impact dans le Monde Réel}
\begin{columns}
\begin{column}{0.5\textwidth}
\begin{block}{Faux Positifs}
\begin{itemize}
\item Ajustements inutiles d'évaluation du corps professoral
\item Adoption de traitements médicaux inefficaces
\item Programmes sociaux mal alloués
\end{itemize}
\end{block}
\end{column}

\begin{column}{0.5\textwidth}
\begin{block}{Faux Négatifs}
\begin{itemize}
\item Associations manquées en santé publique
\item Inégalités sociales non détectées
\item Décisions politiques inefficaces
\end{itemize}
\end{block}
\end{column}
\end{columns}

\begin{alertblock}{Preuves Empiriques}
\begin{itemize}
\item Holt et al. (1980) : $\alpha$ réel variait de 0.05 à 0.50
\item Fay (1985) : "Résultats extrêmement erronés" avec les méthodes SAS
\end{itemize}
\end{alertblock}
\end{frame}

\section{Problématique}
\begin{frame}{Problèmes des tests standards}
\begin{alertblock}{Enjeux des enquêtes complexes}
\begin{itemize}
\item \textbf{Clusters} : Dépendance entre observations (ex: ménages, classes)
\item \textbf{Plan de sondage} : Pondérations et effets de stratification
\item \textbf{Logiciels standards} : Sous-estiment la variance (ex: SAS, SPSS)
\end{itemize}
\end{alertblock}

\begin{exampleblock}{Conséquences}
\begin{itemize}
\item Risque accru d'erreurs de type I (faux positifs)
\item Exemple : $p$-valeurs de 0.03 (SAS) vs 0.25 (réel)
\end{itemize}
\end{exampleblock}
\end{frame}

% Méthode de Wald
\section{Méthode de Wald}
\begin{frame}{Test de Wald pour l'indépendance}
\begin{block}{Formulation}
Pour une table $2 \times 2$ :
\[
\theta = p_{11}p_{22} - p_{12}p_{21}
\]
Estimateur :
\[
\hat{\theta} = \hat{p}_{11}\hat{p}_{22} - \hat{p}_{12}\hat{p}_{21}
\]
Statistique de test :
\[
X_W^2 = \frac{\hat{\theta}^2}{\widehat{Var}(\hat{\theta})} \sim \chi^2_1
\]
\end{block}

\begin{exampleblock}{Enquête sur les jeunes détenus (n=2,588)}
\begin{itemize}
\item SAS : $X^2=11.6$ ($p<0.001$)
\item Méthode des groupes aléatoires : $X_W^2=0.89$ ($p=0.40$)
\end{itemize}
\end{exampleblock}
\end{frame}

% Test de Bonferroni
\section{Test de Bonferroni}
\begin{frame}{Approche de Bonferroni}
\begin{columns}
\begin{column}{0.6\textwidth}
\begin{itemize}
\item Décomposition de $H_0$ en $m$ sous-hypothèses
\item Ajustement du seuil : $\alpha^* = \alpha/m$
\item Comparaison avec une distribution $t$ :
\[
\frac{|\hat{\theta}_{ij}|}{\sqrt{\widehat{Var}(\hat{\theta}_{ij})}} > t_{\kappa}(\alpha/2m)
\]
\end{itemize}
\end{column}

\begin{column}{0.4\textwidth}
\begin{exampleblock}{Exemple Âge/Infraction violente}
\begin{tabular}{ccc}
\toprule
 & $\hat{\theta}_{11}$ & $\hat{\theta}_{12}$ \\
\midrule
Valeur & 0.013 & 0.0119 \\
SE & 0.0074 & 0.0035 \\
$t$ & 1.8 & 3.4 \\
\bottomrule
\end{tabular}
\end{exampleblock}
\end{column}
\end{columns}

\begin{alertblock}{Avantages/limites}
\begin{itemize}
\item Simple à mettre en œuvre
\item Conservateur (risque $\alpha$ contrôlé)
\end{itemize}
\end{alertblock}
\end{frame}

% Corrections des moments
\section{Corrections des moments}
\begin{frame}{Ajustement par les moments}
\begin{block}{Correction de premier ordre (Rao-Scott)}
\[
X_F^2 = \frac{(r-1)(c-1)X^2}{E[X^2]}
\]
où $E[X^2] \approx \sum (1-p_{ij})d_{ij} - ...$
\end{block}

\begin{exampleblock}{Application (Exemple 10.7)}
\begin{itemize}
\item $X^2=34$, $E[X^2]=4.2$
\item $X_F^2 = \frac{2 \times 34}{4.2} = 16.2$
\item Comparaison avec $F_{2,12}$ : $p=0.006$
\end{itemize}
\end{exampleblock}

\begin{block}{Correction de second ordre}
\[
X_S^2 = \frac{\nu X_F^2}{(r-1)(c-1)} \sim \chi^2_\nu
\]
Ajuste aussi la variance (plus précis mais complexe)
\end{block}
\end{frame}

% Méthodes basées sur modèles
\section{Approches modélisées}
\begin{frame}{Modélisation de la dépendance}
\begin{exampleblock}{Exemple des paires de siblings (Cohen 1976)}
\begin{tabular}{lcc}
\toprule
 & Schizophrène (S) & Non (N) \\
\midrule
Hommes & 43 & 15 \\
Femmes & 32 & 52 \\
\bottomrule
\end{tabular}

Modèle avec effet cluster :
\[
q_{ij} = 
\begin{cases} 
\alpha q_i + (1-\alpha)q_i^2 & i=j \\
(1-\alpha)q_i q_j & i\neq j 
\end{cases}
\]
\end{exampleblock}

\begin{alertblock}{Avantages}
\begin{itemize}
\item Capture la dépendance intra-cluster
\item Permet une estimation réaliste des $p$-valeurs
\end{itemize}
\end{alertblock}
\end{frame}

% Conclusion
\section{Conclusion}
\begin{frame}{Recommandations pratiques}
\begin{block}{Bonnes pratiques}
\begin{itemize}
\item Toujours estimer les effets de plan (deff)
\item Préférer les méthodes de rééchantillonnage (bootstrap, jackknife)
\item Utiliser des logiciels spécialisés : R \texttt{survey}, SAS \texttt{PROC SURVEYFREQ}
\end{itemize}
\end{block}

\begin{exampleblock}{Résumé des méthodes}
\begin{tabular}{lp{5cm}}
\toprule
Méthode & Utilisation \\
\midrule
Wald & Tables petites, variance stable \\
Bonferroni & Analyses exploratoires \\
Rao-Scott & Données publiées, deff disponibles \\
Modèles & Structure cluster connue \\
\bottomrule
\end{tabular}
\end{exampleblock}
\end{frame}


\section{Modèles log-linéaires}
\begin{frame}{Modèle d'indépendance}
Pour un tableau \( r \times c \), sous indépendance :
\[
\ln(p_{ij}) = \mu + \alpha_i + \beta_j \quad \text{avec } \sum \alpha_i = \sum \beta_j = 0
\]
\vspace{0.5cm}
\begin{exampleblock}{Exemple : Cable vs Ordinateur}
\begin{table}
\centering
\begin{tabular}{lcc|c}
 & Oui & Non & Total \\ 
\hline 
Cable : Oui & 0.254 & 0.360 & 0.614 \\ 
Non & 0.160 & 0.226 & 0.386 \\ 
\hline 
Total & 0.414 & 0.586 & 1.000 \\ 
\end{tabular}
\end{table}
\end{exampleblock}
\end{frame}

\begin{frame}{Modèle saturé et tests}
\[
\ln(p_{ijk}) = \mu + \alpha_i + \beta_j + \gamma_k + (\alpha\beta)_{ij} + \dots + (\alpha\beta\gamma)_{ijk}
\]
\begin{itemize}
    \item Test d'adéquation : Comparaison via \( X^2 \) ou \( G^2 \).
    \item Enquêtes complexes : Correction des erreurs standards par méthodes de rééchantillonnage (ex: groupes aléatoires).
\end{itemize}

\begin{alertblock}{Attention}
Les logiciels standards (ex: SAS) sous-estiment les variances. Utiliser des packages spécialisés (SUDAAN, WesVarPC).
\end{alertblock}
\end{frame}

%------------------ Section 4: Étude de cas ------------------
\section{Étude de cas : Enquête sur les jeunes détenus}
\begin{frame}{Données et modèle}
Variables : Âge, Antécédents familiaux (FAMTIME), Violence (EVERVIOL).
\begin{table}
\centering
\begin{tabular}{lcccc}
 & \multicolumn{2}{c}{FAMTIME=Non} & \multicolumn{2}{c}{FAMTIME=Oui} \\ 
Âge & Non & Oui & Non & Oui \\ 
\hline 
≤15 & 0.0588 & 0.0698 & 0.0659 & 0.0856 \\ 
16–17 & 0.0904 & 0.1237 & 0.0944 & 0.1375 \\ 
≥18 & 0.0435 & 0.0962 & 0.0355 & 0.0986 \\ 
\end{tabular}
\end{table}
\end{frame}

\begin{frame}{Résultats}
\begin{itemize}
    \item Test de Bonferroni pour les interactions :
    \[
    H_0 : \beta_4 = \beta_5 = \dots = \beta_{11} = 0
    \]
    \item Niveau de signification ajusté : \( \alpha/7 = 0.007 \).
    \item Aucune interaction significative, mais effet marginal pour \textit{âge*famtime}.
\end{itemize}
\vspace{0.5cm}
\includegraphics[width=0.4\textwidth]{interaction_plot}  % Remplacez par une figure appropriée
\end{frame}
% Conclusion
\section{Conclusion}
\begin{frame}{Résumé et Recommandations}
\frametitle{Points Essentiels à Retenir}
\begin{block}{Points Essentiels}
\begin{itemize}
\item Le plan de sondage affecte à la fois les estimations ponctuelles et l'inférence
\item L'échantillonnage en grappes nécessite typiquement un ajustement (deff > 1)
\item La stratification améliore généralement la précision (deff < 1)
\end{itemize}
\end{block}
\begin{block}{Meilleures Pratiques}
\begin{itemize}
\item Toujours incorporer les poids de sondage
\item Utiliser une estimation de variance appropriée au plan
\item Vérifier l'analyse avec des logiciels spécialisés
\end{itemize}
\end{block}
\end{frame}
% Références
\begin{frame}{Références}


\frametitle{Lectures Recommandées}
% Les détails des références restent dans leur langue originale
\begin{thebibliography}{9}

\item Rao, J.N.K. \& Scott, A.J. (1984). "On Chi-squared Tests for Multiway Contingency Tables"
\item Cohen, J.E. (1976). "The Distribution of the Chi-squared Statistic Under Clustered Sampling"

\bibitem{Lohr}
Lohr, S. (2022).
\textit{Sampling: Design and Analysis}.
Cengage Learning.

\bibitem{Agresti}
Agresti, A. (2013).
\textit{Categorical Data Analysis}.
Wiley.

\bibitem{Holt}
Holt, D., Scott, A. J., \& Ewings, P. D. (1980).
Chi-squared tests with survey data.
\textit{Journal of the Royal Statistical Society: Series A}, 143(3), 302-320.
\end{thebibliography}

\begin{center}
\Large Questions ?
\end{center}
\end{frame}

\end{document}