\documentclass[12pt]{beamer}
\usepackage[utf8]{inputenc}
\usepackage[T1]{fontenc}
\usepackage[french]{babel}
\usepackage{amsmath,amssymb}
\usepackage{graphicx}
\usetheme{jambro}

\title[STT 504 Sondage]{Sondage stratifié}
\author{FM, Université Soumaré}
\date{}

\begin{document}

\begin{frame}
\titlepage
\end{frame}

\begin{frame}{Sommaire}
\tableofcontents
\end{frame}

\section{5.1 Principe et justification}
\begin{frame}{5.1 Principe et justification}
\begin{itemize}
    \item Problème du SAS : inefficace si population hétérogène.
    \item Solution : \textbf{stratification} en sous-ensembles homogènes (strates).
    \item Avantages :
    \begin{itemize}
        \item Augmentation de la précision globale.
        \item Comparaisons inter-strates.
        \item Méthodes d'estimation adaptatives (ex: populations nomades/sédentaires).
    \end{itemize}
\end{itemize}
\end{frame}

\section{5.2 Description et notations}
\begin{frame}{5.2 Description et notations}
\textbf{Population :}
\begin{itemize}
    \item Taille totale : $N$
    \item H strates de tailles $N_1, N_2, \dots, N_H$ avec $\sum_{h=1}^H N_h = N$
\end{itemize}

\textbf{Échantillon :}
\begin{itemize}
    \item Taille $n_h$ dans la strate $h$ ($\sum_{h=1}^H n_h = n$)
    \item Taux de sondage : $f_h = \frac{n_h}{N_h}$, $f = \frac{n}{N}$
\end{itemize}
\end{frame}

\begin{frame}{Décomposition de la variance}
Pour $N_h$ grand :
\[
S^2 \approx \underbrace{\sum_{h=1}^H \frac{N_h}{N} S_h^2}_{\text{Dispersion intra-strate}} + \underbrace{\sum_{h=1}^H \frac{N_h}{N} (\bar{Y}_h - \bar{Y})^2}_{\text{Dispersion inter-strate}}
\]
\begin{exampleblock}{Important}
La variance totale dépend de l'hétérogénéité \textbf{intra} et \textbf{inter} strates.
\end{exampleblock}
\end{frame}

\section{5.3 Estimation}
\begin{frame}{5.3 Estimateurs sans biais}
\begin{itemize}
    \item Moyenne par strate : $\bar{y}_h = \frac{1}{n_h} \sum_{i=1}^{n_h} y_{hi}$
    \item Moyenne globale : 
    \[
    \hat{\bar{Y}} = \frac{1}{N} \sum_{h=1}^H N_h \bar{y}_h
    \]
    \item Total global : $\hat{Y} = N\hat{\bar{Y}}$
\end{itemize}
\begin{block}{Preuve (sans biais)}
\[
E(\hat{\bar{Y}}) = \frac{1}{N} \sum_{h=1}^H N_h E(\bar{y}_h) = \bar{Y}
\]
\end{block}
\end{frame}

\section{5.4 Calcul de précision}
\begin{frame}{5.4 Variance de l'estimateur}
\[
V(\hat{\bar{Y}}) = \sum_{h=1}^H \left(\frac{N_h}{N}\right)^2 \left(1 - \frac{n_h}{N_h}\right) \frac{S_h^2}{n_h}
\]
\vspace{0.5cm}
\textbf{Intervalle de confiance :}
\[
\left[ \hat{\bar{Y}} \pm u_{1-\alpha/2} \sqrt{\hat{V}(\hat{\bar{Y}})} \right]
\]
\end{frame}

\section{5.5 Taille de l'échantillon}
\begin{frame}{5.5.1 Allocation proportionnelle}
\begin{itemize}
    \item Taux de sondage uniforme : $n_h = \frac{n}{N} N_h$
    \item Estimateur auto-pondéré : $\hat{\bar{Y}} = \bar{y}$
    \item Variance simplifiée :
    \[
    V(\hat{\bar{Y}}) = \frac{1}{n}\left(1 - \frac{n}{N}\right) \sum_{h=1}^H \frac{N_h}{N} S_h^2
    \]
\end{itemize}
\end{frame}

\begin{frame}{5.5.2 Répartition de Neyman}
\begin{itemize}
    \item Allocation selon la dispersion : 
    \[
    n_h \propto N_h S_h
    \]
    \item Optimal pour minimiser la variance (strates hétérogènes $\rightarrow$ plus grandes tailles).
\end{itemize}
\end{frame}

\begin{frame}{5.5.3 Allocation optimale sous contrainte de coût}
\begin{columns}
\begin{column}{0.6\textwidth}
\[
\begin{cases}
\min \sum_{h=1}^H \left(\frac{N_h}{N}\right)^2 \frac{S_h^2}{n_h} \\
CV = \sum_{h=1}^H c_h n_h
\end{cases}
\]
Solution :
\[
n_h = \frac{N_h S_h / \sqrt{c_h}}{\sum_{h=1}^H N_h S_h / \sqrt{c_h}} \cdot CV
\]
\end{column}
\begin{column}{0.4\textwidth}
\begin{exampleblock}{}
Priorise les strates :
\begin{itemize}
    \item Hétérogènes ($S_h \uparrow$)
    \item Peu coûteuses ($c_h \downarrow$)
\end{itemize}
\end{exampleblock}
\end{column}
\end{columns}
\end{frame}

\end{document}