\documentclass[12pt]{beamer}
\usepackage[utf8]{inputenc}
\usepackage[T1]{fontenc}
\usepackage[french]{babel}
\usepackage{amsmath,amssymb,tabularx}
\usepackage{graphicx}
\usetheme{jambro}

\title[STT 504: Sondage à probabilités inégales]{STT 504 : Sondage à probabilités inégales}
\author{FM, Université Soumaré}
\date{}

\begin{document}

\begin{frame}
\titlepage
\end{frame}

\begin{frame}{Sommaire}
\tableofcontents
\end{frame}

% ==============================================
\section{6.1 Principe et justification}
% ==============================================
\begin{frame}{6.1 Contexte et motivation}
\small
Le sondage aléatoire simple (SAS) suppose une probabilité égale pour toutes les unités. Cette approche devient sous-optimale lorsque :
\begin{itemize}
    \item La population présente une hétérogénéité marquée
    \item Des informations auxiliaires (ex : taille des exploitations) sont disponibles
\end{itemize}

\vspace{0.3cm}
\textbf{Exemple concret :} 
\begin{itemize}
    \item Enquête agricole : Les grandes exploitations contribuent davantage à la production totale. 
    \item Stratégie : Attribuer une probabilité d'inclusion $\pi_\alpha \propto$ superficie cultivée.
\end{itemize}

\begin{block}{Paradoxe apparent}
Une unité fréquemment sélectionnée doit être \textit{moins pondérée} dans l'estimateur pour éviter le biais.\\
Formellement : $w_\alpha = \frac{1}{\pi_\alpha}$ (principe d'Horvitz-Thompson)
\end{block}
\end{frame}

% ==============================================
\section{6.2 Probabilités d'inclusion}
% ==============================================
\begin{frame}{6.2 Définitions fondamentales}
\begin{itemize}
    \item \textbf{Probabilité d'ordre 1 :} 
    \[
    \pi_\alpha = P(\text{unité } \alpha \in \text{échantillon}) \quad \text{avec} \quad \sum_{\alpha=1}^N \pi_\alpha = n
    \]
    
    \item \textbf{Probabilité d'ordre 2 :} 
    \[
    \pi_{\alpha\beta} = P(\alpha \text{ et } \beta \in \text{échantillon}) \quad \text{pour } \alpha \neq \beta
    \]
\end{itemize}

\textbf{Cas particulier du SAS :}
\[
\pi_\alpha = \frac{n}{N} \quad ; \quad \pi_{\alpha\beta} = \frac{n(n-1)}{N(N-1)}
\]

\begin{alertblock}{Contrainte générale}
Pour tout plan à taille fixe $n$ : 
\[
\sum_{\alpha \neq \beta} \pi_{\alpha\beta} = n(n-1)
\]
\end{alertblock}
\end{frame}

% ==============================================
\section{6.3 Estimateurs de Horvitz-Thompson}
% ==============================================
\begin{frame}{6.3.1 Construction de l'estimateur}
\small
Pour un échantillon $S$ tiré selon des probabilités $\pi_\alpha$, l'estimateur de Horvitz-Thompson s'écrit :
\[
\hat{T} = \sum_{\alpha \in S} \frac{Y_\alpha}{\pi_\alpha}
\]
\textbf{Propriété clé :} Sans biais pour le total $T = \sum_{\alpha=1}^N Y_\alpha$ :
\[
E(\hat{T}) = \sum_{\alpha=1}^N \pi_\alpha \cdot \frac{Y_\alpha}{\pi_\alpha} = T
\]

\begin{exampleblock}{Application à la moyenne}
Estimateur de la moyenne populationnelle :
\[
\hat{\bar{Y}} = \frac{1}{N}\hat{T} = \frac{1}{N} \sum_{\alpha \in S} \frac{Y_\alpha}{\pi_\alpha}
\]
\end{exampleblock}
\end{frame}

\begin{frame}{6.3.2 Variance et optimalité}
\scriptsize
La variance de $\hat{T}$ s'exprime par :
\[
V(\hat{T}) = \frac{1}{2} \sum_{\alpha \neq \beta} (\pi_\alpha \pi_\beta - \pi_{\alpha\beta}) \left( \frac{Y_\alpha}{\pi_\alpha} - \frac{Y_\beta}{\pi_\beta} \right)^2
\]

\textbf{Cas idéal :} Si $\pi_\alpha = \lambda Y_\alpha$ (proportionnalité exacte), alors :
\[
\frac{Y_\alpha}{\pi_\alpha} = \frac{1}{\lambda} \quad \forall \alpha \quad \Rightarrow \quad V(\hat{T}) = 0
\]
\vspace{-0.2cm}
\begin{itemize}
    \item En pratique, on utilise une variable auxiliaire $X$ corrélée à $Y$
    \item Choix des probabilités : 
    \[
    \pi_\alpha = n \frac{x_\alpha}{\sum_{\alpha=1}^N x_\alpha}
    \]
\end{itemize}

\begin{block}{Exemple numérique}
Population : $N=4$ exploitations avec $X = [2, 3, 1, 4]$.\\
Pour $n=2$ : $\pi_\alpha = 2 \times \frac{x_\alpha}{10} = [0.4, 0.6, 0.2, 0.8]$
\end{block}
\end{frame}

% ==============================================
\section{6.4 Méthodes de tirage systématique}
% ==============================================
\begin{frame}{6.4.1 Méthode des totaux cumulés}
\footnotesize
\textbf{Étapes détaillées :}
\begin{enumerate}
    \item Ordonner les unités et calculer les tailles cumulées $C_\alpha = \sum_{k=1}^\alpha x_k$
    \item Calculer le pas de tirage : $p = \frac{C_N}{n}$ (ex : $C_N=100$, $n=5$ $\Rightarrow p=20$)
    \item Tirer un nombre aléatoire $u \in [1, p]$
    \item Sélectionner les unités aux positions : $u, u+p, u+2p, \dots$
\end{enumerate}

\textbf{Exemple :}
\begin{columns}
\begin{column}{0.5\textwidth}
Données : \\
\begin{tabular}{|c|c|c|}
\hline
Unité & $x_\alpha$ & Cumul \\
\hline
A & 5 & 5 \\
B & 8 & 13 \\
C & 7 & 20 \\
D & 10 & 30 \\
\hline
\end{tabular}
\end{column}
\begin{column}{0.5\textwidth}
Avec $p=7.5$ et $u=3$ : \\
Unités sélectionnées : B (cumul=13 > 3) et D (cumul=30 > 10.5)
\end{column}
\end{columns}
\end{frame}

\begin{frame}{6.4.2 Méthode des probabilités cumulées}
\footnotesize
\textbf{Algorithme pas-à-pas :}
\begin{enumerate}
    \item Calculer les probabilités d'inclusion $\pi_\alpha$ pour chaque unité
    \item Construire les cumuls : $Q_\alpha = \sum_{k=1}^\alpha \pi_k$
    \item Générer $u \sim \mathcal{U}([0,1])$
    \item Déterminer les seuils : $s_i = u + (i-1)$ pour $i=1,\dots,n$
    \item Sélectionner l'unité $\alpha$ si $Q_{\alpha-1} < s_i \leq Q_\alpha$
\end{enumerate}

\begin{exampleblock}{Application}
Avec $\pi = [0.2, 0.4, 0.3, 0.1]$ et $u=0.15$ :\\
\begin{tabular}{|c|c|c|}
\hline
Unité & Cumul & Sélection \\
\hline
1 & 0.2 & $0.15 \in [0, 0.2]$ $\Rightarrow$ Oui \\
2 & 0.6 & $0.15+1=1.15 \notin [0.2,0.6]$ \\
3 & 0.9 & - \\
4 & 1.0 & - \\
\hline
\end{tabular}
\end{exampleblock}
\end{frame}
\begin{frame}{Synthèse comparative}
\footnotesize
\textbf{Méthode 1 : Totaux cumulés}
\begin{itemize}
    \item[+] Avantages :
    \begin{itemize}
        \item Simplicité conceptuelle
        \item Efficace pour les grandes populations
    \end{itemize}
    \item[--] Limites :
    \begin{itemize}
        \item Sensible à l'ordre des données
        \item Risque de périodicité
    \end{itemize}
\end{itemize}

\vspace{0.5cm}
\textbf{Méthode 2 : Probabilités cumulées}
\begin{itemize}
    \item[+] Avantages :
    \begin{itemize}
        \item Respect exact des probabilités
        \item Adapté aux plans complexes
    \end{itemize}
    \item[--] Limites :
    \begin{itemize}
        \item Calculs intensifs
        \item Nécessite un tri préalable
    \end{itemize}
\end{itemize}


\end{frame}

\begin{frame}{Synthèse comparative}
\footnotesize
\begin{block}{Recommandation}
Privilégier les probabilités cumulées pour les enquêtes de précision, et les totaux cumulés pour des études exploratoires.
\end{block}
\end{frame}

\begin{frame}{Conclusion}
\small
\begin{itemize}
    \item Le sondage à probabilités inégales permet des gains de précision substantiels lorsque des variables auxiliaires sont disponibles
    \item L'estimateur de Horvitz-Thompson fournit un cadre théorique robuste
    \item Le choix de la méthode de tirage dépend du contexte opérationnel et des contraintes calculatoires
\end{itemize}

\vspace{0.5cm}
\begin{block}{Perspectives}
Extension aux plans de sondage multi-étapes et intégration avec la stratification.
\end{block}
\end{frame}

\end{document}