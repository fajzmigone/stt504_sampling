\documentclass{article}
\usepackage{graphicx} % Required for inserting images

\title{correction_term1}
\author{franckmigone }
\date{May 2025}

\begin{document}

\maketitle

\section*{Exercice 1}
\[
\text{CI}_{95\%}
  \;=\;
  \hat{Y} \;\pm\; 1.96
  \sqrt{
    \left(\frac{N-n}{N}\right)\,
    \frac{\hat{S}_y^{2}}{n}}
  \;=\;
  29.07
  \;\pm\; 1.96
  \sqrt{
    \frac{2010-100}{2010}\,
    \frac{707.945}{100}}
  \;=\;
  \bigl[\,23.99,\; 34.15\,\bigr]\;\text{ha}.
\]

\[
\hat{Y}
  = \frac{1}{n}\sum_{k \in S} y_k
  = \frac{2907}{100}
  = 29.07 \text{ ha}.
\]

\[
\hat{S}_{y}^{2}
  = \frac{n}{n-1}
    \left(
      \frac{1}{n}\sum_{k \in S} y_k^{2}
      - \hat{Y}^{\,2}
    \right)
  = \frac{100}{99}
    \left(
      \frac{154\,593}{100}
      - 29.07^{2}
    \right)
  = 707.945.
\]

\section*{Exercice 2}
Une première façon de montrer cette égalité est la suivante :

\[
\frac{1}{2N^{2}}
  \sum_{k \in U}\;
  \sum_{\substack{\ell \in U\\ \ell \neq k}}
  (y_{k} - y_{\ell})^{2}
  \;=\;
  \frac{1}{2N^{2}}
  \sum_{k \in U}\;
  \sum_{\ell \in U}
  (y_{k} - y_{\ell})^{2}
  \;=\;
  \frac{1}{2N^{2}}\!\biggl(
      \sum_{k \in U}\;
      \sum_{\ell \in U} y_{k}^{2}
      + \sum_{k \in U}\;
        \sum_{\ell \in U} y_{\ell}^{2}
      - 2 \sum_{k \in U}\;
        \sum_{\ell \in U} y_{k}y_{\ell}
    \biggr)
\]

\[
= \frac{1}{N}\sum_{k \in U} y_{k}^{2}
  \;-\;
  \frac{1}{N^{2}}
  \sum_{k \in U}\;
  \sum_{\ell \in U} y_{k}y_{\ell}
  \;=\;
  \frac{1}{N}\sum_{k \in U} y_{k}^{2}
  \;-\; \bar{Y}^{\,2}
  \;=\;
  \frac{1}{N}\sum_{k \in U} (y_{k} - \bar{Y})^{2}
  \;=\; \sigma_{y}^{2}.
\]

Une seconde façon est :

\[
\frac{1}{2N^{2}}
  \sum_{k \in U}\;
  \sum_{\substack{\ell \in U\\ \ell \neq k}}
  (y_{k} - y_{\ell})^{2}
  \;=\;
  \frac{1}{2N^{2}}
  \sum_{k \in U}\;
  \sum_{\ell \in U}
  (y_{k} - \bar{Y} - y_{\ell} + \bar{Y})^{2}
\]

\[
= \frac{1}{2N^{2}}
  \sum_{k \in U}\;
  \sum_{\ell \in U}
  \Bigl[(y_{k} - \bar{Y})^{2}
        + (y_{\ell} - \bar{Y})^{2}
        - 2(y_{k} - \bar{Y})(y_{\ell} - \bar{Y})\Bigr]
\]

\[
= \frac{1}{2N}\sum_{k \in U}(y_{k} - \bar{Y})^{2}
  + \frac{1}{2N}\sum_{\ell \in U}(y_{\ell} - \bar{Y})^{2}
  + 0
  \;=\; \sigma_{y}^{2}.
\]

Ainsi, dans les deux démonstrations, on retrouve bien la variance
\(\sigma_{y}^{2}\) de la population.

\medskip
\noindent\textbf{Estimateur sans biais de la variance $\sigma_{y}^{2}$}

L’estimateur sans biais de $\sigma_{y}^{2}$ est défini par
\[
\widehat{\sigma}_{y}^{2}
 \;=\;
 \frac{1}{2N^{2}}
 \sum_{k \in S}\;
 \sum_{\substack{\ell \in S\\ \ell \neq k}}
 \frac{(y_{k} - y_{\ell})^{2}}{\pi_{k\ell}},
\]
où $\pi_{k\ell}$ désigne la probabilité d’inclusion d’ordre 2.  
Dans un plan simple sans remise à taille d’échantillon fixe,
\[
\pi_{k\ell}
 \;=\;
 \frac{n(n-1)}{N(N-1)},
\]
d’où
\[
\widehat{\sigma}_{y}^{2}
 \;=\;
 \frac{N(N-1)}{n(n-1)}
 \,\frac{1}{2N^{2}}
 \sum_{k \in S}\;
 \sum_{\substack{\ell \in S\\ \ell \neq k}}
 (y_{k} - y_{\ell})^{2}.
\]

En adaptant l’égalité (2.1) au seul échantillon $S$ (au lieu de l’univers $U$), on obtient
\[
\frac{1}{2n^{2}}
 \sum_{k \in S}\;
 \sum_{\substack{\ell \in S\\ \ell \neq k}}
 (y_{k} - y_{\ell})^{2}
 \;=\;
 \frac{1}{n}
 \sum_{k \in S} (y_{k} - \hat{Y})^{2},
 \quad\text{où}\quad
 \hat{Y}
 \;=\;
 \frac{1}{n}\sum_{k \in S} y_{k}.
\]

Par conséquent :
\[
\widehat{\sigma}_{y}^{2}
 \;=\;
 \frac{N-1}{N}\;
 \frac{1}{\,n-1\,}
 \sum_{k \in S}(y_{k} - \hat{Y})^{2}
 \;=\;
 \frac{N-1}{N}\;s_{y}^{2}.
\]

Ainsi,
\[
\boxed{\;
  \widehat{\sigma}_{y}^{2}
  \;=\;
  \frac{N-1}{N}\,s_{y}^{2}
  \;}, 
\qquad
\boxed{\;
  \widehat{S}_{y}^{2}
  \;=\;
  \frac{N}{N-1}\,\widehat{\sigma}_{y}^{2}
  \;=\;
  s_{y}^{2}
  \;}.
\]

Ce résultat est classique ; la démonstration serait plus longue sans recourir à l’égalité (2.1).


\end{document}
